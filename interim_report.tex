%------------------------------------------------------------------------------
%   PACKAGES AND OTHER DOCUMENT CONFIGURATIONS
%------------------------------------------------------------------------------

\documentclass[twoside,twocolumn]{article}

\usepackage[english]{babel} % Language hyphenation and typographical rules

% Document margins
\usepackage[hmarginratio=1:1,top=32mm,left=20mm,right=20mm,columnsep=20pt]{geometry}
% Custom captions under/above floats in tables or figures
\usepackage[hang, small,labelfont=bf,up,textfont=it,up]{caption}
\usepackage{booktabs} % Horizontal rules in tables

\usepackage{enumitem} % Customized lists
\setlist[itemize]{noitemsep} % Make itemize lists more compact
\usepackage{textcomp}

% Allows abstract customization
\usepackage{abstract}
% Set the "Abstract" text to bold
\renewcommand{\abstractnamefont}{\normalfont\bfseries}
% Set the abstract itself to small italic text
\renewcommand{\abstracttextfont}{\normalfont\small\itshape}

\usepackage{fancyhdr} % Headers and footers
\pagestyle{fancy} % All pages have headers and footers
\fancyhead{} % Blank out the default header
\fancyfoot{} % Blank out the default footer
\fancyhead[C]{\thetitle}
\fancyfoot[RO,LE]{\thepage} % Custom footer text

\usepackage{titling} % Customizing the title section

\usepackage{hyperref} % For hyperlinks in the PDF
\usepackage{amsmath}

\usepackage{tikz}
\usetikzlibrary{bayesnet}
\usetikzlibrary{arrows}

\usepackage{color}
\usepackage{caption}
\usepackage{subcaption}

\usepackage{graphicx}

\captionsetup[figure]{labelfont={bf},textfont=normalfont}

%------------------------------------------------------------------------------
%   TITLE SECTION
%------------------------------------------------------------------------------

\setlength{\droptitle}{-4\baselineskip} % Move the title up

\pretitle{\begin{center}\Huge\bfseries} % Article title formatting
\posttitle{\end{center}} % Article title closing formatting

\title{MT Interim Report: \\ Neural Word Alignment}
\author{%
\textsc{Bailey Parker} \\[1ex]
\normalsize Johns Hopkins University \\
\normalsize \href{mailto:bailey@jhu.edu}{bailey@jhu.edu}
 \and
 \textsc{Vivian Tsai} \\[1ex]
\normalsize Johns Hopkins University \\
\normalsize \href{mailto:viv@jhu.edu}{viv@jhu.edu}
 \and
  \textsc{William Watson} \\[1ex]
\normalsize Johns Hopkins University \\
\normalsize \href{mailto:billwatson@jhu.edu}{billwatson@jhu.edu}
}

\date{}

%------------------------------------------------------------------------------
\DeclareMathOperator*{\argmax}{arg\,max}
\newcommand{\qdist}[1]{\ifmmode\langle#1\rangle\else\textlangle#1\textrangle\fi}
\renewcommand{\vec}[1]{\mathbf{#1}}
\newlength\mystoreparindent
\newenvironment{myparindent}[1]{%
  \setlength{\mystoreparindent}{\the\parindent}
  \setlength{\parindent}{#1}
  }{%
  \setlength{\parindent}{\mystoreparindent}
}

\begin{document}

% Print the title
\maketitle

% \section{Introduction}

% TODO

%------------------------------------------------------------------------------

\begin{abstract}
% \noindent \blindtext
We will discuss our intial progress made on our Neural Word Alignment.
More specifically, we will discuss our data procurement methods and processing
algorithms. In addition, we elaborate on our current experimental models,
including the Dot Aligner and Bidirectional GRU Aligner. Our loss function,
train/eval implmentation, and batching techiques are also discussed.
\end{abstract}

%------------------------------------------------------------------------------
\section{Introduction}

\section{Data Procurement and Processing}
\subsection{Symbol Tokenization}
\subsection{Number Tokenization}
\subsection{Proper Noun Tokenization}
\subsection{Lemmatization Techniques}
\subsection{POS Tagging}

\section{Model Components}
\subsection{Preliminary Notation}
% inputs, lambdas, scaling facotrs, etc, anything on notation
We begin by stating several operations frequently used in our discussion.

\subsubsection{Hadamard Product}

To perform element-wise multiplication of two matricies $A$ and $B$, of
equivalent dimensions, we use the hadamard product, defined as the $\circ$.
\begin{equation}
  (A \circ B)_{ij} = A_{ij} \cdot B_{ij}
\end{equation}

\subsubsection{Sigmoid Function}

The sigmoid function $\sigma$ is applied element wise and defined as follows:
\begin{equation}
  \sigma(x) = \frac{1}{1+e^{-x}}
\end{equation}

\subsubsection{Hyperbolic Tangent Function}

The hyperbolic tangent function $\tanh$ is defined as follows:
\begin{equation}
  \tanh(x) = \frac{e^x - e^{-x}} {e^x + e^{-x}}
\end{equation}
and is applied element-wise.

\subsubsection{Variables}
We define our source sequence as $s$, target sequence as $t$, and refer to
the $i$-th source token as $s_i$. We refer to the $j$-th target token as $t_j$.
We refer to a matrix $\psi$ whose rows represent values related to source
tokens $s_i$ and whose columns refer to values realted to target tokens $t_j$.
Hence, $\psi$ is an $|s| \times |t|$ sized matrix.

\subsection{Softmax and Log Softmax}
The softmax function transforms a vector of values into a probability
distirbution. Appliying the softmax function to an n-dimensional input tensor
rescales it so that the elements of the n-dimensional output tensor lie in the
range (0,1) and sum to 1.
\begin{equation}
  \sigma(\vec{x})_i = \frac{e^{x_i}}{\sum_j e^{x_j}}
\end{equation}

For a matrix $\psi$, the rows correspond to source words $s_i$ and the
columns correspond to target words $t_j$. In addition, when we softmax
with respect to the targets, i.e. $\sigma_t(\psi)$, the softmax is applied
per column. If applied per row, we denote this as $\sigma_s(\psi)$.
\begin{equation}
  \sigma_t(\psi)_{ij} = \frac{e^{\psi_{ij}}}{\sum_{k} e^{\psi_{kj}}}
\end{equation}
\begin{equation}
  \sigma_s(\psi)_{ij} = \frac{e^{\psi_{ij}}}{\sum_{k} e^{\psi_{ik}}}
\end{equation}

However, it is sometimes better to work in log-space, and use the log
softmax operator for numerical stability.
\begin{equation}
  \log \sigma_t(\psi)_{ij} = \log \frac{e^{\psi_{ij}}}{\sum_{k} e^{\psi_{kj}}}
\end{equation}
\begin{equation}
  \log \sigma_s(\psi)_{ij} = \log \frac{e^{\psi_{ij}}}{\sum_{k} e^{\psi_{ik}}}
\end{equation}

\subsection{Word Embeddings}
Word Embedding layers allow for a simple lookup table that stores embeddings
of a fixed dictionary and size. More specifically, these layers are often used
to store word embeddings and retrieve them using indices. The input to the
embedding layer is a list of indices, and the output is the corresponding
word embeddings. This allows words to be represented numerically to a set
embedding dimension size, and thus can be passed onto later layers.

Word Embeddings can be formulated as a weight matrix $W_e$, where the
vector representation of word $w_i$ is the $i$-th row of the matrix, and
can be represented as
$W_e[w_i]$.

\begin{equation}
  W_e = \begin{bmatrix}
  \longleftarrow w_1 \longrightarrow \\
  \vdots\\
  \longleftarrow w_i \longrightarrow\\
  \vdots\\
  \longleftarrow w_n \longrightarrow
\end{bmatrix}
\end{equation}

\subsection{Gated Recurrent Units (GRU)}
\label{sec:gru}
Gated Recurrent Units are a more complex formulation to recurrent layers to
process sequences, and improve the vanishing gradient problem found in
vanilla RNN layers while using less parameters than a
Long Short-Term Memory (LSTM) layer\cite{grupaper}. A GRU makes use of
3 gates: $r_t$ reset gate; $z_t$ update gate; $n_t$ new gate. These 3 gates
allow for a blending of the hidden state with new input, and are computed
as follows for each input $x_t$ in a sequence $\vec{x}$:
\begin{equation}
  \begin{split}\begin{array}{ll}
  r_t = \sigma(W_{ir} x_t + b_{ir} + W_{hr} h_{t-1} + b_{hr}) \\
  z_t = \sigma(W_{iz} x_t + b_{iz} + W_{hz} h_{t-1} + b_{hz}) \\
  n_t = \tanh(W_{in} x_t + b_{in} + r_t \circ (W_{hn} h_{t-1}+ b_{hn})) \\
  h_t = (1 - z_t) \circ n_t + z_t \circ h_{t-1} \\
  \end{array}\end{split}
\end{equation}
where $x_t$ is input at time $t$, $h_{t-1}$ is the hidden state of the
previous layer at time $t-1$ or the initial hidden state at time $0$, $h_t$
is the new hidden state at time $t$, and $\sigma$ is the sigmoid function.
\subsection{Bidirectional GRU}
\label{sec:bidirectional}

GRUs only process sequences in a forward direction. However, for
translation and in general NLP, we also care about the succeeding input.
%@viv rephrase
We thus introduce the Bidirectional GRU, which that runs two separate GRU
layers in opposite directions and stacks the output s.t. each element has
the context of elements before and after it.

Let us define $\operatorname{GRU}(input)$ as a GRU layer that outputs the
hidden cells from $input$ on a single forward pass. Let us also define the
operator $\overrightarrow{h}$ as the tensor with elements ordered in
order $[0,$ $n]$, and $\overleftarrow{h}$ as the tensor elements ordered
from $[n,$ $0]$, i.e. reversed.

When flipping the arrow, i.e., from $\overrightarrow{h}$ to
$\overleftarrow{h}$, we define this operation as an invert/reverse
function, and it flips the elements to the opposite orientation. We
additionally declare $\Vert$ as the concatenation operator. Finally, we
define $\overrightarrow{h_f}$ as the forward output, $\overleftarrow{h_b}$
as the backward output, and $h_o$ as the final, stacked bidirectional output.

\begin{equation}
  \label{eq:bidirectional}
  \begin{split}
    \begin{array}{ll}
      % fix to have forward and reverse arrows, and update top
      \overrightarrow{h_f} = \operatorname{GRU}(\overrightarrow{input})\\
      \\
      \overleftarrow{h_b} = \operatorname{GRU}(\overleftarrow{input})\\
      \\
      h_o = \overrightarrow{h_f} \,\,\Vert \,\, \overrightarrow{h_b}\\
      \\
    \end{array}
  \end{split}
\end{equation}

From the above equations, we compute the forward output normally;
compute the backwards output on the inverted input; and concatenate
the forward outputs with the inverted backwards outputs. All final
hidden states are provided as well. The output is of
size $(N,$ $batch,$ $hidden$ $size*2)$ since we concatenate the
outputs of the two GRU layers.


\subsection{Alignment Prior}
\label{sec:alingment_prior}
In order to learn diagonal alignemnts, we describe our formulation of the
IBM Model 2 reparameterization as a custom PyTorch layer. We define the
prior alignment matrix $A$ as an $|s| \times |t|$. Given a source token
$s_i$ and target token $t_j$, and alignment hyperparameter $\lambda$, we
define the alignment distortion function $a_{ij}$:
\begin{equation}
  a_{ij} = -\lambda | i - j |
\end{equation}

We set the default $\lambda$ value to $4$. This encodes our diagonal prior,
and allows us to add in the diagonal weights to our network's weights. We
allow for an optional, learnable scaling factor $\alpha$ to control the
strength of this prior, defaulted to $0.25$. Hence our IBM Model 2 module
outputs the alignemnt prior matrix $A$:
\begin{equation}
  A_{ij} = \alpha \cdot a_{ij}
\end{equation}

\subsection{Batch Matrix Multiplication\\(BMM)}
\label{sec:bmm}
To combine source and target tensors, we use Batch Matrix Multiplication.
Given a matrix $A$ of size $(b,$ $n,$ $m)$ and matrix $B$ of size
$(b,$ $m,$ $p)$, we perform matrix multiplication on the sub-matricies
to create an output matrix $O$ of size $(b$ $n,$ $p)$.

\begin{equation}
  O_i = A_i \times B_i
\end{equation}

This allows us to combine the vectorized output of our network on each
individual sequence and combine them into a matrix tensor of
size $(b,$ $|s|,$ $|t|)$. This requires the use of simple transpose
functions to align the dimensions of each pairing sentence to correctly
compute the BMM.

\section{Model Descriptions}
We use the aformentioned model components to construct complex alignment
models, and provide a more through discussion on our techiques.
\subsection{Dot Aligner}
This is our baseline model. This model attempts to learn word embeddings
to correctly give alignemnt outputs.

The intuition behind this baseline mdoel was to compute dot products
between every source token $s_i$ embedding and target token $t_j$ embedding
to generate an alignment matrix via BMM (Section ~\ref{sec:bmm}), combined
with the diagonal prior discussed in Section ~\ref{sec:alingment_prior}.
The ouput is the alignment matrix $\psi$.

This is a very simple model that only learns the pure embeddings, and
therefore we do not expect it to do as well as more complex models.
The full computation graph can be seen in Figure ~\ref{fig:dot_aligner}.

\begin{figure}
  \centering
  \begin{tikzpicture}

    % Define nodes
    \node[latent] (plus) {$+$};
    \node[latent, below=1cm of plus] (psi) {$\psi$};
    \node[rectangle, above=0.75cm of plus, xshift=-2.25cm] (bmm) {BMM};
    \node[rectangle, above=0.5cm of bmm, xshift=-1cm] (embed_s) {Embed};
    \node[rectangle, above=0.5cm of bmm, xshift=1cm] (embed_t) {Embed};
    \node[obs, above=0.75cm of embed_s] (s) {$s$};
    \node[obs, above=0.75cm of embed_t] (t) {$t$};
    \node[obs, right=0.6cm of t] (lamb) {$\lambda$};
    \node[obs, right=0.6cm of lamb] (i) {$i$};
    \node[obs, right=0.6cm of i] (j) {$j$};
    \node[rectangle, below=0.75cm of i] (align) {$a_{ij} = -\lambda |i-j|$};
    % \node[rectangle, below=1cm of align] (logsoft) {$\log \sigma_t(a)$};
    \node[latent, below=0.75cm of align] (mux) {$\times$};
    \node[latent, right=0.5cm of align] (alpha) {$\alpha$};


    % Connect the nodes
    \edge {s} {embed_s};
    \edge {t} {embed_t};
    \edge {embed_s, embed_t} {bmm};
    \edge {lamb, i, j} {align};
    % \edge {align} {logsoft};
    \edge {align, alpha} {mux};
    \edge {mux, bmm} {plus};
    \edge {plus} {psi};

  \end{tikzpicture}
  \caption{Model Architecture for Dot Aligner for source $s$,
  target $t$ alignments. $\alpha$ is a global learnable scaling factor for
  the importance of alignment distribution $a$, and $\lambda$ is the
  global alignment distortion parameter.}
  \label{fig:dot_aligner}
\end{figure}

\subsection{Bidirectional GRU Aligner}

An extension to the capacity of the Dot Aligner is to add a
bidirectional GRU after the embedding layers. This will allow the
model to consider not only forward propagation of the input sequence,
but also the reverse. Therefore, our vectorized encodings of each
sequence has the context of the words in its local area for context.
The downstream algorithm for forward propagation is the same after
the recurrent layers, as detailed in Figure ~\ref{fig:gru_aligner}.

\begin{figure}
  \centering
  \begin{tikzpicture}

    % Define nodes
    \node[latent] (plus) {$+$};
    \node[latent, below=1cm of plus] (psi) {$\psi$};
    \node[rectangle, above=0.5cm of plus, xshift=-2.25cm] (bmm) {BMM};
    \node[rectangle, above=0.5cm of bmm, xshift=-1cm] (gru_s) {GRU};
    \node[rectangle, above=0.5cm of bmm, xshift=1cm] (gru_t) {GRU};
    \node[rectangle, above=0.5cm of gru_s] (embed_s) {Embed};
    \node[rectangle, above=0.5cm of gru_t] (embed_t) {Embed};
    \node[obs, above=0.75cm of embed_s] (s) {$s$};
    \node[obs, above=0.75cm of embed_t] (t) {$t$};
    \node[obs, right=0.6cm of t] (lamb) {$\lambda$};
    \node[obs, right=0.6cm of lamb] (i) {$i$};
    \node[obs, right=0.6cm of i] (j) {$j$};
    \node[rectangle, below=0.75cm of i] (align) {$a_{ij} = -\lambda |i-j|$};
    % \node[rectangle, below=1cm of align] (logsoft) {$\log \sigma_t(a)$};
    \node[latent, below=1cm of align] (mux) {$\times$};
    \node[latent, right=0.5cm of align] (alpha) {$\alpha$};


    % Connect the nodes
    \edge {s} {embed_s};
    \edge {t} {embed_t};
    \edge {embed_s} {gru_s};
    \edge {embed_t} {gru_t};
    \edge {gru_s, gru_t} {bmm};
    \edge {lamb, i, j} {align};
    % \edge {align} {logsoft};
    \edge {align, alpha} {mux};
    \edge {mux, bmm} {plus};
    \edge {plus} {psi};

  \end{tikzpicture}
  \caption{Model Architecture for Bidirectional GRU Aligner
  for source $s$, target $t$ alignments. $\alpha$ is a global
  learnable scaling factor for the importance of alignment
  distribution $a$, and $\lambda$ is the global alignment
  distortion parameter.}
  \label{fig:gru_aligner}
\end{figure}

\subsection{Extensions}
Here we discuss possible extensions to the two aformentioned models,
to improve the capacity to learn alignments based upon our initial results.

\section{Loss Function}
% quick tldr, point to appendix for each sub thing, jsut describe
% it
We describe two modes of training: Supervised and Unsupervised.
Supervised training allows us to measure the capactiy of our models
to learn given alignments, and is the first step is validating our
models ability to eventually learn unsupervised alignments.

\subsection{Supervised Loss}
We can formulate our supervised loss as maximizing the probabilities
of our alignments, $\psi$, with the ground truth alignments,
represented as $\phi$.

Our targets are binary valued alignment matricies of size $|s| \times |t|$.
Our targets $\phi$ are defined as follows for a source token $s_i$ and
target token $t_j$:
% get better notation
\begin{equation}
  \phi_{ij} = \begin{cases}
  1 & \text{if } s_i \text{ aligns with } t_j \\
  0 & \text{else}
  \end{cases}
\end{equation}

Hence, if a source token $s_i$ aligns with $t_j$, our target is 1.
We now formulate our loss objective as the maximum likelihood estimate (MLE)
between our alignment matrix $\psi$ and target matrix $\phi$.
However, $\psi$ are hard weights, and we use the softmax function to
convert our alignment weights to valid probabilities. In addition, we
seek to maximize the probability for both the row ($\sigma_s$) and
column ($\sigma_t$) softmax. Hence our MLE function to be maximized is:

\begin{equation}
  MLE(\psi, \phi) = \prod_i \prod_j \left[ \frac{\exp \psi_{ij}}{\sum_k \exp \psi_{ik}} \right]^{\phi_{ij}} \cdot \left[ \frac{\exp \psi_{ij}}{\sum_k \exp \psi_{kj}} \right]^{\phi_{ij}}
\end{equation}


\begin{equation}
  MLE(\psi, \phi) = \prod_i \prod_j \left[ \sigma_s(\psi)_{ij} \right]^{\phi_{ij}} \cdot \left[ \sigma_t(\psi)_{ij} \right]^{\phi_{ij}}
\end{equation}

We can see that by maximizing the MLE function allows our network to
drive the probabilities of alignment to be as close to $1$ as possible,
and implicity depresses all other values along the row and column.
However, neural networks do not maximizing objective functions, and
the multiplication of small probaiblites leads to issues of numerical
stabliltiy. We therefore define the Negative Log Likelihood function:

\begin{equation}
  NLL(\psi, \phi) = - \sum_i \sum_j  \phi_{ij} \cdot \log \sigma_s (\psi)_{ij} + \phi_{ij} \cdot \log \sigma_t (\psi)_{ij}
\end{equation}

We sum up the log alignment probabilities corresponding to our
target alignments, and this is minimized when the probability of alignment
is high, and hence we can learn target alignments between source
sentence $s$ and target sentence $t$.

\begin{figure}
  \centering
  \begin{tikzpicture}

    % Define nodes
    \node[latent] (loss) {$\mathcal{L}$};
    \node[rectangle, above=0.5cm of loss] (nsum) {$-\sum$};
    \node[obs, above=2.5cm of nsum] (mask) {$\phi$};
    \node[obs, above=1cm of mask] (psi) {$\psi$};
    \node[latent, above=0.5cm of nsum, xshift=-0.75cm] (muxs) {$\times$};
    \node[latent, above=0.5cm of nsum, xshift=0.75cm] (muxt) {$\times$};
    \node[rectangle, above=0.5cm of muxs, xshift=-0.75cm] (logs) {$\log \sigma_s (\psi)$};

    \node[rectangle, above=0.5cm of muxt, xshift=0.75cm] (logt) {$\log \sigma_t (\psi)$};

    % Connect the nodes
    \edge {psi} {logs, logt};
    \edge {logs, mask} {muxs};
    \edge {logt, mask} {muxt};
    \edge {muxs, muxt} {nsum};
    \edge {nsum} {loss};

  \end{tikzpicture}
  \caption{Computation graph for our supervised loss function.
    The generated alignment matrix $\psi$ is compared to the ground
    truth alignment matrix $\phi$ to output a loss value $\mathcal{L}$.}
  \label{fig:supervised_loss}
\end{figure}

\subsection{Unsupervised Alignment Loss}

% refrence appendix

Our unsupervised loss function, to be minimized, is a 5-term equation.
We define:
\begin{itemize}[label={}]
  \item $a_t$ as the target prior alignment matrix normalized per column with respect to $t$\\ %per col wrt t
  \item $a_s$ as the source prior alignment matrix normalized per row with respect to $s$\\ %per row wrt s
  \item $\sigma_s$ as the softmax operator applied on the rows of a matrix\\
  \item $\sigma_t$ as the softmax operator applied to each column of a matrix
\end{itemize}

Indexing is done under the assumption that source words $s_i$ form rows
and target words $t_j$ form columns. The term $ij$ is a shorthand
for $s_i$ and $t_j$ indexing into our matrices.

\begin{equation}
  \centering
\begin{split}
  Lo&ss = \\
  &- \sum_j^{|t|} \log \left[
      \sum_i^{|s|} \exp \left(
        \log \sigma_s(\theta(t, s))_{ij} + \log \sigma_t(\psi)_{ij} \right)
    \right] \\
  &+ \sum_i^n \sum_j^m \sigma_t(\psi)_{ij} \cdot \log \left[
    \frac{\sigma_t(\psi)_{ij}}{a_t(i, j)} \right] \\
  &- \sum_i^{|s|} \log \left[ \sum_j^{|t|}
      \exp \left(
        \log \sigma_t(\theta(t, s))_{ij} + \log \sigma_s(\psi)_{ij}
      \right)
    \right] \\
  &+ \sum_i^n \sum_j^m \sigma_s(\psi)_{ij} \cdot \log \left[
    \frac{\sigma_s(\psi)_{ij}}{a_s(i, j)} \right] \\
  &- \log \sum_i^{|s|} \sum_j^{|t|} \left[
    \sigma_s(\psi) \circ \sigma_t(\psi) \right]_{ij} \\
\end{split}
\end{equation}

The full derivation of the loss function and its component terms
can be found in Appendix ~\ref{appendix:loss-function}.


\section{Train/Eval Implmentation}
\subsection{Vocabulary Building}
\subsection{Batching}
\subsection{Optimizer}
Adam
\subsection{Alignment Generations}
Our models generate alignment matricies $\psi$. However, these matricies
are just weights, and we need to covnert them into meanigful alignments.
Hence we describe several methods to covert our weights into actual alignemnts.

\subsubsection{Argmax Alignments}
The simplest and naive version of generating alignments is to
align a source token $s_i$ to a target token $t_j$ such that the
alignment weight $\psi$ is maximized. We can do this with respect
to the source (row) or target (column).
\begin{equation}
  r_{ij} = \begin{cases}
    1 & \text{if } j=\argmax_k \psi_{ik} \\
    0 & \text{else}
  \end{cases}
\end{equation}

\begin{equation}
  c_{ij} = \begin{cases}
    1 & \text{if } i=\argmax_k \psi_{kj} \\
    0 & \text{else}
  \end{cases}
\end{equation}

However, this forces a single alignment for a row or column.

\subsubsection{Union, Intersection, and Grow-Diag-Final}

An improvement to the naive argmax alignments is to consider the
union and intersection of the alignments generated. The union would
be the element-wise hadamard product of the row and column alignments.

\begin{equation}
  u_{ij} = r_{ij} \cdot c_{ij}
\end{equation}

This method generates alignments that both directions agree on.
However, this is a more conservative alignement method that either
argmax method alone. Hence, we could take theu intersection.
This generates alignments where only one of the original directions
has to align.

\begin{equation}
  n_{ij} = r_{ij} \text{ or } c_{ij}
\end{equation}

% Add grow diag final here.


\subsubsection{Thresholding}

The best approach would be to allow for alignments to be generated
under a threshold scheme. Hence we can generate multiple alignments
per row (or column) that are likely. We can then apply the
Grow-Diag-Final method to prune alignments to the union plus
ajacent intersections.


We first standardize our alingment weights to have zero mean and
standard deviation of 1. This bounds the thresholding cirteria,
since our weights can get arbitrailly large.

For a source (row) based standardization:

\begin{equation}
  \mu_{i} = \frac{1}{|t|} \sum_j \psi_{ij}
\end{equation}
\begin{equation}
  \sigma_{i} = \sqrt{\frac{\sum_j \left( \psi_{ij} - \mu_i \right)^2}{|t|-1}}
\end{equation}
\begin{equation}
  Z_{ij} = \frac{\psi_{ij} - \mu_i}{\sigma_i}
\end{equation}

For a target (column) standardization, we compute means and
standard deviations along the columns.

We can then apply a threshold to the values to generate alignments.
% find adaptive threshold methods
\begin{equation}
  a_{ij} = \begin{cases}
    1 & \text{if } Z_{ij} > \tau \\
    0 & \text{else}
  \end{cases}
\end{equation}


\section{Ground Truth Alignment Results}
% ?

\section{Current Status}
% what do we currently have for him?

\section{Future Work}
% what to get done by final report



%------------------------------------------------------------------------------

% References
\bibliographystyle{abbrv}
\bibliography{interim_report}

%------------------------------------------------------------------------------


\clearpage
\appendix
\onecolumn
\begin{myparindent}{0pt}
\section{Loss Function}
\label{appendix:loss-function}
To create our loss function, we convert our alignment matrix $\psi$
to a probability distribution. We define $\sigma(\vec{x})$ as
the softmax operator applied on vector $\vec{x}$, and $\sigma(\vec{x})_i$ as
the $x_i$ softmax probability for vector $\vec{x}$.

\begin{equation}
\sigma(\vec{x})_i = \frac{\exp(x_i)}{\sum_j\exp(x_j)}
\end{equation}

We therefore define two operations on the alignment matrix $\psi$. For source
$s$ to target $t$ probability generations, we define $\sigma_t(\psi)$ as the
softmax on each column, i.e., target word $t_j$.
\begin{equation}
  \sigma_t(\psi) = \left[
    \begin{matrix}
      \sigma(\psi_{t_1}) &
      \hdots &
      \sigma(\psi_{t_j}) &
      \hdots &
      \sigma(\psi_{t_m})  \\
    \end{matrix}
\right]
\end{equation}
Equivalently,

\begin{equation}
  \sigma_t(\psi)_{ij} = \frac{e^{\psi_{ij}}}{\sum_{k} e^{\psi_{kj}}}
\end{equation}

In addition, we further define the target-to-source generation as
$\sigma_s(\psi)$, where the softmax operator is applied on each row for target
word $s_i$ and produces a row vector.
\begin{equation}
  \sigma_s(\psi) = \left[
    \begin{matrix}
      \sigma(\psi_{s_1})  \\
      \vdots \\
      \sigma(\psi_{s_i})  \\
      \vdots \\
      \sigma(\psi_{s_n})  \\
    \end{matrix}
\right]
\end{equation}
Equivalently,

\begin{equation}
  \sigma_s(\psi)_{ij} = \frac{e^{\psi_{ij}}}{\sum_{k} e^{\psi_{ik}}}
\end{equation}

While VAEs sample from the probability distribution derived by the neural
network, our model's distribution is discrete. Therefore, we need not
sample and can instead compute the loss in terms of all possibilities.

\subsection{Probability Maximization}

In our original formulation, we
maximized probabilities for word alignments.
For this model, we define $p(t, s | \theta, \psi)$ as such:

\begin{equation}
  p(t, s | \theta, \psi)
    = \prod_j^{|t|} \sum_i^{|s|} p_\theta(t_j| s_i) \cdot p_a(i|j)
    = \prod_j^{|t|} \sum_i^{|s|} \sigma_s(\theta(t_j, s_i)) \cdot \sigma_t(\psi)_{ij}
\end{equation}

This formula stems from our assumption that the alignments for each
target word are conditionally independent, hence a product over target
words $t_j$. We then sum over all the alignment possibilities from the source
words $s_i$, thus marginalizing over the alignments.

For $s \mapsto t$ alignments, alignments are softmaxed per column because
we hold the target $t_j$ constant and iterate
over the source $s_i$, hence $\sigma_t$.

The translation probabilities are softmaxed per row (i.e., source word $s_i$)
since we are given a source word and want to know the probability of
translating $s_i$ to $t_j$, hence $\sigma_s$.

However, as we do not maximize in neural networks, we must transform this
equation into a minimization problem by taking the negative log of $p$:

\begin{equation}
  -\log p(t, s | \theta, \psi) =
  - \sum_j^{|t|}
     \log \left[ \sum_i^{|s|} \sigma_s \left( \theta(t_j, s_i) \right) \cdot
      \sigma_t(\psi)_{ij} \right]
\end{equation}

We can simplify this into the equation below, for easier implementation
(via PyTorch's \texttt{logsumexp} function):

\begin{equation}
  -\log  p(t , s | \theta, \psi) =
  - \sum_j^{|t|}  \log \left[ \sum_i^{|s|} \exp
      \left( \log \sigma_s(\theta(t_j, s_i)) + \log \sigma_t(\psi)_{ij} \right)
    \right]
\end{equation}

This is the term to be minimized for our source-to-target word alignment
generation.


\subsection{KL Divergence and Prior Terms}

We also need to minimize the Kullback-Leibler (KL) divergence between our
distribution and a prior. For discrete probability distributions $P$ and $Q$
defined on the same probability space, the reverse KL divergence from $Q$ to
$P$ is defined as:

\begin{equation}
D_{\mathrm{KL}}(Q \| P) = \sum_{i} Q(i) \log \left( \frac{Q(i)}{P(i)} \right)
\end{equation}

In other words, it is the expectation of the log difference between the
probabilities $P$ and $Q$, where the expectation is taken using the
probabilities $Q$.

To calculate the KL divergence of our model, we must define
distributions $P$ and $Q$. We first consider $P$ as our prior
distribution, henceforth called $a_t$. This prior distribution $a_t$ is a
$n \times m$ matrix filled with the alignment probabilities filled for source
word $s_i$; target word $t_j$; source sentence $s$ of length $n$; and target
sentence $t$ of length $m$:

We define distortion exponent $h$ as:
\begin{equation}
  h(i, j) = {-\lambda \left| \frac{i}{n} - \frac{j}{m}\right|}
\end{equation}

% might have to flip this to be sum over i' not j' DONE
\begin{equation}
  Z_j = \sum_{i'} \exp h(i', j)
\end{equation}

\begin{equation}
a_t (i, j) =
\begin{cases}
      p_0 & \text{if } null \\
     (1-p_0) \cdot \frac{e^{h(i,j)}}{Z_j} & \text{else}
   \end{cases}
\end{equation}

In \cite{dyer2013simple}, parameter values were selected as
$\lambda = 4$ and $p_0 = 0.08$ for the entire corpus.
Each element value of $a_t$ is normalized by the sum of the column distortion
values to create a valid distribution via term $Z_j$
(since we hold target $t_j$ constant). For future notation, let the subscript
on the distribution $a$ denote the way we normalize; i.e., $a_t$ is the
prior alignment distribution normalized with respect to target
words $t_j$ and thus normalized by the sum of the column for $t_j$.

% \todo: change?
We can then write our KL Divergence (to be minimized) as:

\begin{equation}
  D_{\mathrm{KL}}(\sigma_t(\psi) \| a_t) =
    \sum_i^n \sum_j^m \sigma_t(\psi)_{ij} \cdot
      \log \left[ \frac{\sigma_t(\psi)_{ij}}{a_t(i, j)} \right]
\end{equation}


\subsection{Target-to-Source Loss Terms}

We have described the loss terms for a source $s$ to target $t$ word alignment.
However, we seek to perform alignment by agreement. Hence, we must add loss
functions that describe the evaluation of target $t$ to source $s$:

\begin{equation}
  -\log  p(t , s | \theta, \psi) =
  - \sum_i^{|s|}  \log \left[ \sum_j^{|t|}
      \exp \left(
        \log \sigma_t(\theta(t_j, s_i)) + \log \sigma_s(\psi)_{ij}
      \right)
    \right]
\end{equation}

\noindent
Additionally, for the KL Divergence term for target $t$ to source $s$:

\begin{equation}
D_{\mathrm{KL}} (\sigma_s(\psi) \| a_s) = \sum_i^n \sum_j^m \sigma_s(\psi)_{ij}
  \cdot \log \left[ \frac{\sigma_s(\psi)_{ij}}{a_s(i, j)} \right]
\end{equation}

\noindent
where the normalization of the prior $a_s$ is on the row instead of the column.
In other words, we hold the source $s_i$ constant; sum the row; and divide each
row element by the sum.

The equations for $a_s$ are as follows:

\begin{equation}
  h(i, j) = {-\lambda \left| \frac{i}{n} - \frac{j}{m}\right|}
\end{equation}

% might have to flip this to be sum over i' not j' DONE
\begin{equation}
  Z_i = \sum_{j'} \exp h(i, j')
\end{equation}

\begin{equation}
a_s (i, j) =
\begin{cases}
      p_0 & \text{if } null \\
     (1-p_0) \cdot \frac{e^{h(i,j)}}{Z_i} & \text{else}
   \end{cases}
\end{equation}

We also perform the softmax operator on each column of the $\psi$ matrix,
previously defined as $\sigma_t(\psi)$.


\subsection{Alignment by Agreement}

Finally, one last loss function term must be added to jointly train each model.
We define $\circ$ as the Hadamard Product, which is the element-wise
multiplication of two matrices. For instance:
$(A \circ B)_{ij} = A_{ij} \cdot B_{ij}$.

We can write the term as such:

\begin{equation}
  -\log \sum_i^{|s|} \sum_j^{|t|}
    \left[ \sigma_s(\psi) \circ \sigma_t(\psi) \right]_{ij}
\end{equation}

% Again note here, you can use logsumexp trick to implement this efficiently in
% PyTorch


\subsection{Combined Loss Function}

Our final loss function, to be minimized, is a 5-term equation. We define:
\begin{itemize}[label={}]
  \item $a_t$ as the target prior alignment matrix normalized per column with respect to $t$\\ %per col wrt t
  \item $a_s$ as the source prior alignment matrix normalized per row with respect to $s$\\ %per row wrt s
  \item $\sigma_s$ as the softmax operator applied on the rows of a matrix\\
  \item $\sigma_t$ as the softmax operator applied to each column of a matrix
\end{itemize}

Indexing is done under the assumption that source words $s_i$ form rows
and target words $t_j$ form columns. The term $ij$ is a shorthand
for $s_i$ and $t_j$ indexing into our matrices.

\begin{equation}
  \centering
\begin{split}
  Lo&ss = \\
  &- \sum_j^{|t|} \log \left[
      \sum_i^{|s|} \exp \left(
        \log \sigma_s(\theta(t, s))_{ij} + \log \sigma_t(\psi)_{ij} \right)
    \right] \\
  &+ \sum_i^n \sum_j^m \sigma_t(\psi)_{ij} \cdot \log \left[
    \frac{\sigma_t(\psi)_{ij}}{a_t(i, j)} \right] \\
  &- \sum_i^{|s|} \log \left[ \sum_j^{|t|}
      \exp \left(
        \log \sigma_t(\theta(t, s))_{ij} + \log \sigma_s(\psi)_{ij}
      \right)
    \right] \\
  &+ \sum_i^n \sum_j^m \sigma_s(\psi)_{ij} \cdot \log \left[
    \frac{\sigma_s(\psi)_{ij}}{a_s(i, j)} \right] \\
  &- \log \sum_i^{|s|} \sum_j^{|t|} \left[
    \sigma_s(\psi) \circ \sigma_t(\psi) \right]_{ij} \\
\end{split}
\end{equation}

% Note, the alignment prior is not exactly the same for source-to-target and
% target-to-source because even though the loss function is symmetric, the
% probabilities are normalized with respect to n or m, depending which
% direction you are translating
\end{myparindent}
\end{document}

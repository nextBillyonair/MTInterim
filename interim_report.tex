%------------------------------------------------------------------------------
%   PACKAGES AND OTHER DOCUMENT CONFIGURATIONS
%------------------------------------------------------------------------------

\documentclass[twoside,twocolumn]{article}

\usepackage[english]{babel} % Language hyphenation and typographical rules

% Document margins
\usepackage[hmarginratio=1:1,top=32mm,left=20mm,right=20mm,columnsep=20pt]{geometry}
% Custom captions under/above floats in tables or figures
\usepackage[hang, small,labelfont=bf,up,textfont=it,up]{caption}
\usepackage{booktabs} % Horizontal rules in tables

\usepackage{enumitem} % Customized lists
\setlist[itemize]{noitemsep} % Make itemize lists more compact
\usepackage{textcomp}

% Allows abstract customization
\usepackage{abstract}
% Set the "Abstract" text to bold
\renewcommand{\abstractnamefont}{\normalfont\bfseries}
% Set the abstract itself to small italic text
\renewcommand{\abstracttextfont}{\normalfont\small\itshape}

\usepackage{fancyhdr} % Headers and footers
\pagestyle{fancy} % All pages have headers and footers
\fancyhead{} % Blank out the default header
\fancyfoot{} % Blank out the default footer
\fancyhead[C]{\thetitle}
\fancyfoot[RO,LE]{\thepage} % Custom footer text

\usepackage{titling} % Customizing the title section

\usepackage{hyperref} % For hyperlinks in the PDF
\usepackage{amsmath}

\usepackage{tikz}
\usetikzlibrary{bayesnet}
\usetikzlibrary{arrows}

\usepackage{color}
\usepackage{caption}
\usepackage{subcaption}

\usepackage{graphicx}

\captionsetup[figure]{labelfont={bf},textfont=normalfont}

%------------------------------------------------------------------------------
%   TITLE SECTION
%------------------------------------------------------------------------------

\setlength{\droptitle}{-4\baselineskip} % Move the title up

\pretitle{\begin{center}\Huge\bfseries} % Article title formatting
\posttitle{\end{center}} % Article title closing formatting

\title{MT Interim Report: \\ Neural Word Alignemnt}
\author{%
\textsc{Bailey Parker} \\[1ex]
\normalsize Johns Hopkins University \\
\normalsize \href{mailto:bailey@jhu.edu}{bailey@jhu.edu}
 \and
 \textsc{Vivian Tsai} \\[1ex]
\normalsize Johns Hopkins University \\
\normalsize \href{mailto:viv@jhu.edu}{viv@jhu.edu}
 \and
  \textsc{William Watson} \\[1ex]
\normalsize Johns Hopkins University \\
\normalsize \href{mailto:billwatson@jhu.edu}{billwatson@jhu.edu}
}

\date{}

%------------------------------------------------------------------------------
\DeclareMathOperator*{\argmax}{arg\,max}
\newcommand{\qdist}[1]{\ifmmode\langle#1\rangle\else\textlangle#1\textrangle\fi}
\renewcommand{\vec}[1]{\mathbf{#1}}

\begin{document}

% Print the title
\maketitle

% \section{Introduction}

% TODO

%------------------------------------------------------------------------------

\begin{abstract}
% \noindent \blindtext
% Add tldr on interim
\end{abstract}

%------------------------------------------------------------------------------
\section{Introduction}

\section{Data Procurement and Processing}
\subsection{Symbol Tokenization}
\subsection{Number Tokenization}
\subsection{Proper Noun Tokenization}
\subsection{Lemmatization Techniques}
\subsection{POS Tagging}

\section{Model Components}
\subsection{Preliminary Notation}
% inputs, lambdas, scaling facotrs, etc, anything on notation
We begin by stating several operations frequently used in our discussion.

\subsubsection{Hadamard Product}

To perform element-wise multiplication of two matricies $A$ and $B$, of equivalent dimensions, we use the hadamard product, defined as the $\circ$.
\begin{equation}
  (A \circ B)_{ij} = A_{ij} \cdot B_{ij}
\end{equation}

\subsubsection{Sigmoid Function}

The sigmoid function $\sigma$ is applied element wise and defined as follows:
\begin{equation}
  \sigma(x) = \frac{1}{1+e^{-x}}
\end{equation}

\subsubsection{Hyperbolic Tangent Function}

The hyperbolic tangent function $\tanh$ is defined as follows:
\begin{equation}
  \tanh(x) = \frac{e^x - e^{-x}} {e^x + e^{-x}}
\end{equation}
and is applied element-wise.

\subsubsection{Variables}
We define our source sequence as $s$, target sequence as $t$, and refer to the $i$-th source token as $s_i$. We refer to the $j$-th target token as $t_j$. We refer to a matrix $\psi$ whose rows represent values related to source tokens $s_i$ and whose columns refer to values realted to target tokens $t_j$. Hence, $\psi$ is an $|s| \times |t|$ sized matrix.

\subsection{Softmax and Log Softmax}
The softmax function transforms a vector of values into a probability distirbution. Appliying the softmax function to an n-dimensional input tensor rescales it so that the elements of the n-dimensional output tensor lie in the range (0,1) and sum to 1.
\begin{equation}
  \sigma(\vec{x})_i = \frac{e^{x_i}}{\sum_j e^{x_j}}
\end{equation}

For a matrix $\psi$, the rows correspond to source words $s_i$ and the columns correspond to target words $t_j$. In addition, when we softmax with respect to the targets, i.e. $\sigma_t(\psi)$, the softmax is applied per column. If applied per row, we denote this as $\sigma_s(\psi)$.
\begin{equation}
  \sigma_t(\psi)_{ij} = \frac{e^{\psi_{ij}}}{\sum_{k} e^{\psi_{kj}}}
\end{equation}
\begin{equation}
  \sigma_s(\psi)_{ij} = \frac{e^{\psi_{ij}}}{\sum_{k} e^{\psi_{ik}}}
\end{equation}

However, it is sometimes better to work in log-space, and use the log softmax operator for numerical stability.
\begin{equation}
  \log \sigma_t(\psi)_{ij} = \log \frac{e^{\psi_{ij}}}{\sum_{k} e^{\psi_{kj}}}
\end{equation}
\begin{equation}
  \log \sigma_s(\psi)_{ij} = \log \frac{e^{\psi_{ij}}}{\sum_{k} e^{\psi_{ik}}}
\end{equation}

\subsection{Word Embeddings}
Word Embedding layers allow for a simple lookup table that stores embeddings of a fixed dictionary and size. More specifically, these layers are often used to store word embeddings and retrieve them using indices. The input to the embedding layer is a list of indices, and the output is the corresponding word embeddings. This allows words to be represented numerically to a set embedding dimension size, and thus can be passed onto later layers.

Word Embeddings can be formulated as a weight matrix $W_e$, where the vector representation of word $w_i$ is the $i$-th row of the matrix, and can be represented as
$W_e[w_i]$.

\begin{equation}
  W_e = \begin{bmatrix}
  \longleftarrow w_1 \longrightarrow \\
  \vdots\\
  \longleftarrow w_i \longrightarrow\\
  \vdots\\
  \longleftarrow w_n \longrightarrow
\end{bmatrix}
\end{equation}

\subsection{Gated Recurrent Units (GRU)}
Gated Recurrent Units are a more complex formulation to recurrent layers to process sequences, and improve the vanishing gradient problem found in vanilla RNN layers while using less parameters than a Long Short-Term Memory (LSTM) layer\cite{grupaper}. A GRU makes use of 3 gates: $r_t$ reset gate; $z_t$ update gate; $n_t$ new gate. These 3 gates allow for a blending of the hidden state with new input, and are computed as follows for each input $x_t$ in a sequence $\vec{x}$:
\begin{equation}
  \begin{split}\begin{array}{ll}
  r_t = \sigma(W_{ir} x_t + b_{ir} + W_{hr} h_{t-1} + b_{hr}) \\
  z_t = \sigma(W_{iz} x_t + b_{iz} + W_{hz} h_{t-1} + b_{hz}) \\
  n_t = \tanh(W_{in} x_t + b_{in} + r_t \circ (W_{hn} h_{t-1}+ b_{hn})) \\
  h_t = (1 - z_t) \circ n_t + z_t \circ h_{t-1} \\
  \end{array}\end{split}
\end{equation}
where $x_t$ is input at time $t$, $h_{t-1}$ is the hidden state of the previous layer at time $t-1$ or the initial hidden state at time $0$, $h_t$ is the new hidden state at time $t$, and $\sigma$ is the sigmoid function.
\subsection{Bidirectional GRU}
\subsection{Alignment Prior}
\subsection{Batch Matrix Multiplication (BMM)}

\section{Model Descriptions}
\subsection{Dot Aligner}
\subsection{Bidirectional GRU Aligner}
\subsection{Extensions}

\section{Loss Function}
% quick tldr, point to appendix for each sub thing, jsut describe
% it
\subsection{Supervised Loss}
\subsection{Unsupervised Alignment Loss}

\section{Ground Truth Alignment Results?}
% ?

\section{Current Status}
% what do we currently have for him?

\section{Future Work}
% what to get done by final report



%------------------------------------------------------------------------------

% References
\bibliographystyle{abbrv}
\bibliography{interim_report}

%------------------------------------------------------------------------------


\clearpage
\appendix
\onecolumn

\end{document}
